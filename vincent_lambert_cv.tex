%%%%%%%%%%%%%%%%%
% This is an sample CV template created using altacv.cls
% (v1.7.2, 28 August 2024) written by LianTze Lim (liantze@gmail.com). Compiles with pdfLaTeX, XeLaTeX and LuaLaTeX.
%
%% It may be distributed and/or modified under the
%% conditions of the LaTeX Project Public License, either version 1.3
%% of this license or (at your option) any later version.
%% The latest version of this license is in
%%    http://www.latex-project.org/lppl.txt
%% and version 1.3 or later is part of all distributions of LaTeX
%% version 2003/12/01 or later.
%%%%%%%%%%%%%%%%

%% Use the "normalphoto" option if you want a normal photo instead of cropped to a circle
% \documentclass[10pt,a4paper,normalphoto]{altacv}

\documentclass[10pt,a4paper,ragged2e,withhyper]{altacv}
%% AltaCV uses the fontawesome5 and simpleicons packages.
%% See http://texdoc.net/pkg/fontawesome5 and http://texdoc.net/pkg/simpleicons for full list of symbols.

% Change the page layout if you need to
\geometry{left=1.25cm,right=1.25cm,top=1.5cm,bottom=1.5cm,columnsep=1.2cm}

% The paracol package lets you typeset columns of text in parallel
\usepackage{paracol}

% Change the font if you want to, depending on whether
% you're using pdflatex or xelatex/lualatex
% WHEN COMPILING WITH XELATEX PLEASE USE
% xelatex -shell-escape -output-driver="xdvipdfmx -z 0" sample.tex
\iftutex 
  % If using xelatex or lualatex:
  \setmainfont{Roboto Slab}
  \setsansfont{Lato}
  \renewcommand{\familydefault}{\sfdefault}
\else
  % If using pdflatex:
  \usepackage[rm]{roboto}
  \usepackage[defaultsans]{lato}
  % \usepackage{sourcesanspro}
  \renewcommand{\familydefault}{\sfdefault}
\fi

% Change the colours if you want to
\definecolor{SlateGrey}{HTML}{2E2E2E}
\definecolor{LightGrey}{HTML}{666666}
\definecolor{DarkPastelBlue}{HTML}{082645}  %450808
\definecolor{PastelBlue}{HTML}{0D4C8F} %8F0D0D
\definecolor{LightBlue}{HTML}{6195CD} %8F0D0D
\definecolor{GoldenEarth}{HTML}{E7D192}
\colorlet{name}{black}
\colorlet{tagline}{PastelBlue}
\colorlet{heading}{DarkPastelBlue}
\colorlet{headingrule}{GoldenEarth}
\colorlet{subheading}{PastelBlue}
\colorlet{tech}{LightBlue}
\colorlet{accent}{PastelBlue}
\colorlet{emphasis}{SlateGrey}
\colorlet{body}{LightGrey}

% Change some fonts, if necessary
\renewcommand{\namefont}{\Huge\rmfamily\bfseries}
\renewcommand{\personalinfofont}{\footnotesize}
\renewcommand{\cvsectionfont}{\LARGE\rmfamily\bfseries}
\renewcommand{\cvsubsectionfont}{\large\bfseries}


% Change the bullets for itemize and rating marker
% for \cvskill if you want to
\renewcommand{\cvItemMarker}{{\small\textbullet}}
\renewcommand{\cvRatingMarker}{\faCircle}
% ...and the markers for the date/location for \cvevent
% \renewcommand{\cvDateMarker}{\faCalendar*[regular]}
% \renewcommand{\cvLocationMarker}{\faMapMarker*}


% If your CV/résumé is in a language other than English,
% then you probably want to change these so that when you
% copy-paste from the PDF or run pdftotext, the location
% and date marker icons for \cvevent will paste as correct
% translations. For example Spanish:
% \renewcommand{\locationname}{Ubicación}
% \renewcommand{\datename}{Fecha}


%% Use (and optionally edit if necessary) this .tex if you
%% want to use an author-year reference style like APA(6)
%% for your publication list
% % When using APA6 if you need more author names to be listed
% % because you're e.g. the 12th author, add apamaxprtauth=12
% \usepackage[backend=biber,style=apa6,sorting=ydnt]{biblatex}
% \defbibheading{pubtype}{\cvsubsection{#1}}
% \renewcommand{\bibsetup}{\vspace*{-\baselineskip}}
% \AtEveryBibitem{%
%   \makebox[\bibhang][l]{\itemmarker}%
%   \iffieldundef{doi}{}{\clearfield{url}}%
% }
% \setlength{\bibitemsep}{0.25\baselineskip}
% \setlength{\bibhang}{1.25em}


%% Use (and optionally edit if necessary) this .tex if you
%% want an originally numerical reference style like IEEE
%% for your publication list
\usepackage[backend=biber,style=numeric,sorting=ydnt,defernumbers=true]{biblatex}
%% For removing numbering entirely when using a numeric style
\setlength{\bibhang}{1.25em}
% \DeclareFieldFormat{labelnumberwidth}{\makebox[\bibhang][l]{\itemmarker}}
\setlength{\biblabelsep}{0pt}
\defbibheading{pubtype}{\cvsubsection{#1}}
\renewcommand{\bibsetup}{\vspace*{-\baselineskip}}
\AtEveryBibitem{%
  \makebox[\bibhang][l]{~}%
  \iffieldundef{doi}{}{\clearfield{url}}%
}


%% publications.bib contains your publications
\addbibresource{publications.bib}
% \usepackage{academicons}\let\faOrcid\aiOrcid
\begin{document}


\name{Vincent Lambert}
\tagline{Ingénieur et doctorant en informatique}
%% You can add multiple photos on the left or right
\photoR{2.8cm}{photo_cv}
% \photoL{2.5cm}{Yacht_High,Suitcase_High}

\personalinfo{%
  % Not all of these are required!
  \email{vincent.lambert29@gmail.com}
  \location{14 rue des déportés du 11 novembre 1943, 38100 Grenoble, France}
  % \phone{+336 47 54 36 73}
  \homepage{vincent-lambert.gitlab.io/website/}
  \github{Vincent-LAMBERT}
  \orcid{0009-0002-7983-0949}
  %% You can add your own arbitrary detail with
  %% \printinfo{symbol}{detail}[optional hyperlink prefix]
  % \printinfo{\faPaw}{Hey ho!}[https://example.com/]

  %% Or you can declare your own field with
  %% \NewInfoFiled{fieldname}{symbol}[optional hyperlink prefix] and use it:
  % \NewInfoField{gitlab}{\faGitlab}[https://gitlab.com/]
  % \gitlab{your_id}
  %%
  %% For services and platforms like Mastodon where there isn't a
  %% straightforward relation between the user ID/nickname and the hyperlink,
  %% you can use \printinfo directly e.g.
  % \printinfo{\faMastodon}{@username@instace}[https://instance.url/@username]
  %% But if you absolutely want to create new dedicated info fields for
  %% such platforms, then use \NewInfoField* with a star:
  % \NewInfoField*{mastodon}{\faMastodon}
  %% then you can use \mastodon, with TWO arguments where the 2nd argument is
  %% the full hyperlink.
  % \mastodon{@username@instance}{https://instance.url/@username}
}

\makecvheader
%% Depending on your tastes, you may want to make fonts of itemize environments slightly smaller
% \AtBeginEnvironment{itemize}{\small}

%% Set the left/right column width ratio to 6:4.
\columnratio{0.6}

% Start a 2-column paracol. Both the left and right columns will automatically
% break across pages if things get too long.
\begin{paracol}{2}
\cvsection{Formation}

\cvevent{Doctorat en Interaction Humain-Machine}{Équipe IIHM du Laboratoire d'Informatique de Grenoble}{Septembre 2022 -- Actuel}{Saint Martin d'Hères, France}

Sujet de thèse: « Découvrabilité et représentation des interactions par microgestes de la main ». Publications: \cite{lambert_simultaneous_2024}, \cite{lambert_visual_2023}, \cite{lambert_doctoral_consortium}.

\begin{itemize}
\item Extensions \tech{Inkscape} et logiciels d'expérience en \tech{Python} (Inkex, Mediapipe, OpenCv), application de smartwatch (\tech{Kotlin} avec \tech{Android Studio}), site statique (\tech{HTML/CSS/JavaScript}) et dynamique (\tech{NodeJs/Express})
\end{itemize}

\divider

\cvevent{Diplôme d’ingénieur ENSIMAG}{Grenoble INP - Ensimag}{Septembre 2019 -- Juillet 2022}{Saint Martin d'Hères, France}

Formation d’ingénieur en Informatique et Mathématiques appliquées avec une
spécialisation en Ingénierie des Systèmes d’Information.
\begin{itemize}
\item Principaux projets en \tech{Java}, \tech{C}, \tech{SQL} et \tech{HTML/CSS/JavaScript}
\item Petits projets en \tech{Python}, \tech{Rust}, \tech{VHDL}
\end{itemize}

\divider

\cvevent{CPGE PCSI-PSI}{Lycée Paul Cézanne}{Septembre 2017 -- Juillet 2019}{Aix-en-Provence, France}

Préparation intensive aux concours d’entrée en école d’ingenieur.

\divider

\cvevent{Baccalauréat S mention Très Bien}{Externat Saint Joseph}{Septembre 2017}{Ollioules, France}

Diplôme obtenu au sein de la section SI, option Latin.

\medskip

\cvsection{Parcours professionnel}

\cvevent{Projet de fin d’études (stage de 6 mois)}{Université Grenoble Alpes}{21 Février 2022 -- 22 Juillet 2022}{Saint Martin d'Hères, France}
Sujet de stage: «Discoverability des micro-gestes en Réalité Augmentée». Publication \cite{lambert_visual_2023} écrite pendant la thèse sur les expériences menées lors de ce stage.
\begin{itemize}
  \item Images 2D (\tech{Gimp}, \tech{Inkscape}), modélisation 3D (\tech{Blender}), application en Réalité Augmentée (\tech{Unity}), site statique (\tech{HTML}, \tech{CSS}, \tech{JavaScript})
\end{itemize}

\divider

\cvevent{Stage Assistant Ingénieur (2-3 mois)}{Metrologic Group}{25 Mai 2021 -- 06 Août 2021}{Meylan, France}
Sujet de stage: «Portabilité Windows-Linux». Conversion automatique des solutions \tech{Visual Studio} en projets \tech{CMake} (fait en \tech{Python}).

% \cvsection{A Day of My Life}

% % Adapted from @Jake's answer from http://tex.stackexchange.com/a/82729/226
% % \wheelchart{outer radius}{inner radius}{
% % comma-separated list of value/text width/color/detail}
% \wheelchart{1.5cm}{0.5cm}{%
%   6/8em/accent!30/{Sleep,\\beautiful sleep},
%   3/8em/accent!40/Hopeful novelist by night,
%   8/8em/accent!60/Daytime job,
%   2/10em/accent/Sports and relaxation,
%   5/6em/accent!20/Spending time with family
% }

% use ONLY \newpage if you want to force a page break for
% ONLY the current column

\medskip

\cvsection{Enseignement}

\cvevent{R212 Intégration}{IUT 1 Grenoble - L1}{A venir}{}
Suivi de TP et d'examen (20 heqTD).

\divider

\cvevent{R111 Intégration}{IUT 1 Grenoble - L1}{08 Octobre 2024 -- 03 Décembre 2024}{}
Suivi de TP et d'examen (22 heqTD).

\divider

\cvevent{Interaction multimodale et sur supports mobiles}{Polytech Grenoble - M2}{13 Septembre 2022 -- 15 Novembre 2022}{}
Cours, suivi de projet et soutenances (41 heqTD).

\divider

\cvevent{Cours Experience Utilisateur (UX)}{Université Grenoble Alpes - M2}{28 Septembre 2022 -- 07 Décembre 2022}{}
Suivi de projet (16,5 heqTD).

\divider

\cvevent{Interaction multimodale et sur supports mobiles}{Polytech Grenoble - M2}{14 Septembre 2022 -- 16 Novembre 2022}{}
Suivi de projet et soutenances (31 heqTD).

\divider

\cvevent{Cours Experience Utilisateur (UX)}{Université Grenoble Alpes - M2}{28 Septembre 2022 -- 07 Décembre 2022}{}
Suivi de projet (16,5 heqTD) et examen de 1h (rédaction de sujet et correction).

\medskip

\cvsection{Publications}

%% Specify your last name(s) and first name(s) as given in the .bib to automatically bold your own name in the publications list.
%% One caveat: You need to write \bibnamedelima where there's a space in your name for this to work properly; or write \bibnamedelimi if you use initials in the .bib
%% You can specify multiple names, especially if you have changed your name or if you need to highlight multiple authors.
\mynames{Lim/Lian\bibnamedelima Tze,
  Wong/Lian\bibnamedelima Tze,
  Lim/Tracy,
  Lim/L.\bibnamedelimi T.}
%% MAKE SURE THERE IS NO SPACE AFTER THE FINAL NAME IN YOUR \mynames LIST

\nocite{*}

\printbibliography[heading=pubtype,title={\printinfo{\faBook}{Books}},type=book]

\divider

\printbibliography[heading=pubtype,title={\printinfo{\faFile[regular]}{ Articles de Journal}},type=article]

\divider

\printbibliography[heading=pubtype,title={\printinfo{\faUsers}{ Conference Proceedings}},type=inproceedings]

\medskip

\cvsection{Projets personnels majeurs}

\cvevent{Créateur des règles, de l'univers et des outils}{Jeu de rôle papier}{Août 2015 -- en cours}{}
Création d’un jeu de rôle papier avec système de règles, lore, bêta-tests et quêtes (fichiers compilés avec \tech{LuaLaTeX} et une base de données \tech{MariaDB}). Création d’une langue avec
grammaire, vocabulaire, système d’écriture et donc fonte pour les fichiers numériques associés. Réalisation d’une
interface graphique avec \tech{JavaFx} pour l’exploration du jeu, une modification aisée de ses variables et pour la création assistée de personnages.

\divider

\cvevent{Game designer, scénariste et aspects techniques}{Modification de jeux existants (hack-rom)}{Janvier 2014 -- Août 2017}{}
Création de multiples hack-roms (jeux modifiés) basés sur l’univers de Pokémon.

\medskip

\cvsection{Autres intérêts}

% Adapted from @Jake's answer from http://tex.stackexchange.com/a/82729/226
% \wheelchart{outer radius}{inner radius}{
% comma-separated list of value/text width/color/detail}
\wheelchart{1.5cm}{0.5cm}{%
  6/8em/accent!30/Jeu de rôle,
  3/8em/accent!40/Handball,
  8/8em/accent!60/{Jeu-vidéo,\\e-sport},
  2/10em/accent/Cuisine vegan,
  5/6em/accent!20/Manga
}

\medskip

%% Switch to the right column. This will now automatically move to the second
%% page if the content is too long.
\switchcolumn

% \cvsection{My Life Philosophy}

% \begin{quote}
% ``Something smart or heartfelt, preferably in one sentence.''
% \end{quote}

\cvsection{Fiertés}

\cvachievement{\faTrophy}{Mon Jeu de Rôle}{10 ans de travail pour créer un univers complet et un système de jeu original}

\divider

\cvachievement{\faHeartbeat}{Mes engagements associatifs}{3 associations pendant le covid}

\divider

\cvachievement{\faHeartbeat}{Ma thèse}{manuscrit en cours d'écriture...}

% \cvsection{En quelques mots}

% % Don't overuse these \cvtag boxes — they're just eye-candies and not essential. If something doesn't fit on a single line, it probably works better as part of an itemized list (probably inlined itemized list), or just as a comma-separated list of strengths.

% % The `ragged2e` document class option might cause automatic linebreaks between \cvtag to fail.
% % Either remove the ragged2e option; or 
% % add \LaTeXraggedright in the paragraph for these \cvtag
% {\LaTeXraggedright
% \cvtag{Créatif}
% \cvtag{Polyvalent}
% \cvtag{Moteur}
% \par}

% \medskip

% Et niveau perso...
% \smallskip

% {\LaTeXraggedright
% \cvtag{Jeux-vidéo}
% \cvtag{E-sport}
% \cvtag{Jeu de rôle}
% \cvtag{Manga}
% \par}

\cvsection{Langues}

\cvskill{Français}{français.png}{Langue maternelle}
\smallskip
\divider

\cvskill{Anglais}{anglais.png}{C1 (TOEIC) et B2 (First)}
\smallskip
\divider

\cvskill{Espagnol}{espagnol.png}{B2 (d’après le CECR)}
\smallskip
\divider

%% Yeah I didn't spend too much time making all the
%% spacing consistent... sorry. Use \smallskip, \medskip,
%% \bigskip, \vspace etc to make adjustments.
\medskip

\cvsection{Associations}

\cvlightevent{Association des Faluchards Grenoblois}{Président}{Septembre 2021 -- Septembre 2022}{}
Gestion d’évènements étudiants (soirées trimestrielles et weekend spécial annuel a \textasciitilde150 personnes), de
commandes groupées et de partenariats locaux.

\divider

\cvlightevent{Pôle Communication de l’Ensimag}{Président}{Avril 2020 -- Mai 2021}{}

Écriture pour le journal d’école mensuel et prise de photographies/vidéos.

\divider

\cvlightevent{Bureau des Arts (BDA) de l’Ensimag}{Vice-trésorier}{Avril 2020 -- Mai 2021}{}

Budgets prévisionnel de l’année (Solde de trésorerie \textasciitilde4500€). Organisation de deux soirées Quiz en distanciel (\textasciitilde60
personnes sur Discord pendant 4h).

\divider

\cvlightevent{Ensimagaming}{Reponsable communication}{Mai 2020 -- Décembre 2020}{}

Gestion en binôme de la communication. Mise en place de la chaine Twitch de l’association. Organisation et
commentaire de tournoi Rocket League.

\divider

\cvlightevent{Puls’Art – Liste BDA Ensimag 2020}{Vice-trésorier}{Octobre 2019 -- Avril 2020}{}

Budgets prévisionnel de la semaine d’évènements culturels (Total \textasciitilde8000€).

\medskip

\cvsection{Encadrement}

\cvsmallevent{Jad	Berjawi (M1)}{}{12 Février 2024  --	 23 Août 2024}{}
Sujet de stage: « Develop a Unity AR plugin for hand microgestures ».

\divider

\cvsmallevent{Raphaël Demoulin (L3)}{}{30 Mai 2023  -- 28 Juillet 2023}{}
Sujet de stage: « Programmation d'une API pour la représentation de
microgestes de la main ».

\medskip

\cvsection{Autres expériences}

\cvsmallevent{}{Séminaire LOKI}{3 juin 2024 -- 5 juin 2024}{Lille, France et Hastière, Belgique}
3 jours de présentations et discussions sur les travaux de recherche de l'équipe LOKI.

\divider

\cvsmallevent{}{PHD LIG Madness}{21 Mai 2024}{Saint Martin d'Hères, France}
Présentation de ma thèse en 2 minutes. Récompensé avec 2 autres doctorants.

\divider

\cvsmallevent{}{eNSEMBLE Autumn School}{05 Octobre 2023 -- 07 Octobre 2023}{Saint Martin d'Hères, France}
Trois jours de formations en Interaction Humain-Machine avec des pairs venant d’universités françaises (\textasciitilde20h).

\divider

\cvsmallevent{}{Etudiant volontaire IHM 2023}{04 Avril 2023 –- 08 Avril 2023}{Troyes, France}
Gestion technique du comité de programme (\textasciitilde7h).
Gestion du micro, logistique à l’UTT et logistique du gala (\textasciitilde15h).


\newpage

\cvsmallevent{}{Formation SAFE}{24 Février 2022}{En ligne}
Formation au soutien face aux agressions sexistes et sexuelles (VSS). Vidéos, discussions et questionnaire incluant des aspects statistiques, sociologiques, psychologiques et juridiques (\textasciitilde5h au total). Sensibilisation à l'accompagnement des victimes et auteur·rice·s de VSS.


\divider

\cvsmallevent{}{AFIRM CHI Summer School}{22 Août 2022 –- 26 Août 2022}{Padoue, Italie}
Formations à l’Interaction Humain-Machine avec des pairs (\textasciitilde25h).


% \divider

% \cvsection{Referees}

% % \cvref{name}{email}{mailing address}
% \cvref{Prof.\ Alpha Beta}{Institute}{a.beta@university.edu}
% {Address Line 1\\Address line 2}

% \divider

% \cvref{Prof.\ Gamma Delta}{Institute}{g.delta@university.edu}
% {Address Line 1\\Address line 2}


\end{paracol}


\end{document}
